\documentclass[11pt]{article}

\usepackage{xepersian}

\title{\lr{F-Bridge Development}}
\author{پرهام الوانی}

\begin{document}
	\begin{titlepage}
		\begin{center}
			\emph{به نام خدا}
		\end{center}
		\maketitle
	\end{titlepage}
	\tableofcontents
	\newpage
	\section{\lr{T-REX}}
	ماژول پراکسی برای پروتکل \lr{REST} مبتنی بر \lr{C} و کتابخانه متن باز \lr{libsoup} تحت عنوان \lr{T-REX} پیاده سازی شد و تحت چند سناریو ساده امتحان شد. 
	در ادامه با موفقیت آمیز بودن این آزمایشها این مازول به \lr{Pipeline} اضافه شد.
	\section{\lr{TRAP}}
	این ماژول یک پیاده ساری اولیه تک نخی برای انجام گام های \lr{Pipeline} است. این ماژول وظیفه ی توزیع و فراخوانی سایر ماژولها در \lr{Pipeline} را بر عهده دارد.
	\section{\lr{Ryu}}
	مطالعات اندکی روی \lr{Ryu SDN Controller} ها به منظور بررسی صلاحیتشان برای استفاده در پروژه انجام شد که طی آن یک نمونه کد تحت عنوان \lr{L2 Switch} برای تست این پلتفرم نوشته شد. در ادامه \lr{REST API} این پلتفرم 
	به منظور بازبینی و افزودن ویژگی های مورد نیاز برای \lr{Local Bridge Agent} در مخزن گیت قرار گرفت و دستورات آن دسته بندی و طبقه بندی شد.
	\section{\lr{Open-VZ}}
	پلتفرم تست \lr{Application} به همراه ویژگی \lr{live migration} به وسیله ی دو ماشین مجازی \lr{ubuntu} راه اندازی شد و تست گردید. در ادامه با توجه به بررسی های انجام شده مشخص شد که \lr{Open-VZ} هیچگونه \lr{API} ای ندارد و برای کار کردن با آن
	میبایست از دستورات آن استفاده نمود. به این ترتیب برای دسترسی به بار هر یک از \lr{container} ها میبایست ار همین دستورات یا از برنامه هایی مانند \lr{zenoss} استفاده نمود.
\end{document}